\section{Mergers in the universe}
\textbf{Mergers are important phenomena since they occur frequently throughout cosmic history and contribute to the shaping of structures observed today. When a galaxy first forms or when galaxies merge together they are out of equilibrium. During the structure formation and evolution of the universe, mergers of galaxies and other objects happens frequently. Such mergers cause phase mixing and violent relaxation which are both collisionless relaxation processes. They drive each other when a system undergoes collapse; e.g. during galaxy formation} \\ \\

\subsection{Phase mixing}
The CBE is satisfied during galaxy mergers. Phase mixing is the process where regions of lower phase space density is mixed together with regions of higher phase space density. It is important when a galaxy relaxes towards a steady state, and it works by changing the coarse-grained phase-space density near the phase point of each star. Phase mixing and the CBE has the combined effect of making the maximum value of the DF decrease in a monotonic way. The systems mass fraction positioned at DF values above any value is reduced. [3]

\subsection{Violent relaxation}
Violent relaxation changes the energy of each star in a given galaxy. 
When the gravitational potential $\Phi (x,t)$ of a given galaxy depends both on position and time, The corresponding energy is changing with time in the following way: \\
\begin{equation}
\frac{d E}{d t} = \frac{\partial \Phi}{\partial t} \bigg|_{x(t)}
\end{equation}
This is what Simulation type II tries to do by providing a kick-flow pattern of energy exchange and subsequent drift.
Violent relaxation widens the range of energies for the particles. Whether the energy of a given particle will increase or decrease depends on starting position and velocity but is independent of the particles own mass. This makes collision-less systems significantly different from gas systems where equipartition of energy happens as time goes and a gas system reaches thermal equilibrium. Mergers of galaxies or halos is simulated by type I. More on this in a later section.

\textbf{With the understanding of these effects we are prepared to set up simulations that mimic these effects. First we will see some density models in the following section which accurately describes dark halos and subsequently use some of these density models together with other parameter choices to create a large set of both stable and unstable Initial Conditions in the following section.}