\section{Conclusion}
The $\gamma + \kappa$ vs. $\beta$-plots gives useful insight to the effects of each different simulation types for various ICs. \\
\textbf{Sim. I} clearly reproduced the attractor found earlier [7]. The main difference between this sim. and the one done by others is the increased number of perturbation->simulation repetitions as well as the removal of the few gravitationally unbound particles residing in the outer regions of structures toward the later stages of the simulations. The more numerous experiments done in this way indicates a true attractor which is robust against repeated perturbations->simultions as any true attractor must be (as well as any stable points).  \\ \\

End products of \textbf{sim. $II_a$} landed on a almost vertical curve indicating that this type of sim. drives structures closer to isotropy. This behavior is in agreement with previous results [6]. The structures does not become completely isotropic but are attracted towards a state in the Jeans parameter space where $\beta = 0.1$ in the outer parts. This thus indicate the presence of a new attractor. The reason for this probably resides in the fact that this sim. type is symmetric in velocity kicks whereas previous work had been symmetric in energy kicks (perturbing kinetic energy in the range [0.25,1.75]) [4]. This subtle difference might lead to differences in the VDFs which might then dictate the outcome. \\ \\

\textbf{Sim. $II_b$} does not result in any attractor, as the parturbations and flow have no systematic effect on structures and they are seen to fill up a larger and larger span of the parameter space between $\beta$, $\gamma$ and $\kappa$. Perhaps this is just an artificial type of simulation which does not accurately resemble real physical procecces in the universe. One could try to find systematic features in the flow by providing smaller velocity kicks than the ones here utilized. This is in agreement with [4]. \\ \\

End products of \textbf{sim. $II_c$} landed on the characteristic s-shaped attractor curve. This is in agreement with [4]. \\ \\

End products of \textbf{sim. $II_d$} landed on the same almost vertical curve as the one found in $II_a$ indicating that this type of sim. similarly drives structures closer to isotropy. This is the same outcome as the type $II_a$ sim., which is to be expected as it is basically a subclass of $II_a$. This behavior (approaching isotropy) is in agreement with previous results [6]. Again the new attractor seems present in the outer parts of the end products. \\ \\

A relationship between the velocity anisotropy, the radial slope of the density profile, and the radial slope of the squared velocity dispersion is thus confirmed for stable configurations when undergoing simulations of type I (G-perturbations) and $II_c$ (perturbing tangential velocities).
This relationship exists in the form of a one-dimensional attractor (s-shaped curve). 

\centerline{\textit{concluding the side projects (See appendix B),}} 
\centerline{\textbf{cosmological simulations}} 
The cosmological analysis of dark halos morphology indicates that halo shape generally has slight departures from sphericity and can be either oblate or prolate in cosmological simulations. \\

\centerline{\textbf{VDF}} 
Velocity profiles (or Velocity Distribution Functions) of dark matter halos are seen to depart slightly from Gaussian distributions and to be better fitted by Tsallis power laws (Generalized Gaussian distributions) which are more steep and have flatter tails. \\

\centerline{\textbf{LOS}} 
A comparison is made between 3D structures and their corresponding line-of-sight (LOS) appearance showing that details in fine-structure is lost when going from the former to the latter. This is a complication related to observations that does not exist for simulations where a complete 3D view can be obtained.

\centerline{\textbf{Bumpy road to universalities}} 
The radial slope of the density profile is seen to converge towards a characteristic shape with 6 extrema, i.e. 3 maxima and 3 minima. To analyse these features in more detail, each extremum can be applied in turn as a normalization to various structures to see their agreement with, or departure from this, in a closer look. \\ \\

All-in-all an attractor is found alongside other universal trends and it is therefore concluded that the results of structure formation and evolution is not completely random but follow deep underlying physical principles. There must be some physical reason behind this type of behavior as it has been found for a large span of ICs undergoing different types of sims. resembling physical processes taking place during galaxy mergers. The velocity distributions of DM needs to be understood in more detail or understanding the effects of violent relaxation and phase mixing better might give new clues. Hansen, Juncher and Sparre (2010) suggests that this attractor is fundamental for spherical dark matter structures and that it might be seen as a fundamental relation for dark matter; just as the NFW-profile (maybe the attractor can help in explaning why the NFW profile is universal.) and the $\frac{\rho}{\sigma^3}$-relation. The attractor is at least one more interesting feature of our universe concerning dark matter, which might lead to new understanding in the cosmological context. Putting some of the pieces together, we can get closer to solving the Jeans eq.; The density is now modelled, and we have the knowledge of the pseudo-phase-space density as well. Putting these two pieces of information together, this fixes $\sigma_{rad}^2(r)$. In order to determine $\beta$ it still remains to find the tangential velocity dispersion, $\sigma_{tan}^2(r)$. Non-cosmological studies has been conducted to explore the properties of $\beta$-profiles further,e.g. searching for a linear relation between $\beta$ and $\gamma$, which is found in the inner region of cosmological halos. Mass estimates of large structures using the J.E. (derived for a steady spherical system without bulk flow) can be off by up to 40 $\%$ when falsely assuming $\beta = 0$ in some cases. This M-$\beta$ degeneracy (the mass will be different for different choices of $\beta$) could be solved if $\beta$ is known. So for dwarf galaxies for example if $\sigma_r$ and $\rho$ is found then after applying the attractor curve $\beta$ will be known as well thus potentially eliminating the M-$\beta$ degeneracy completely. The J.E. thus effectively removes one degree of freedom from the J.E. \\ 

The advancements in thermodynamics driven by Maxwell and Boltzmann gave a microscopic understanding of universal trends between macroscopic quantities such as pressure, temperature and density for a ideal gas, 
namely by $P = T \cdot \rho$ which lead to the invention of modern statistical mechanics. \\
In terms of DM, finding the relationships between the macroscopic quantities such as density $\rho$, velocity dispersion squared $\sigma^2$ and the velocity anisotropy parameter $\beta$ is where we are at today. Searching for the underlying microscopic physics in terms of VDF's (radial and tangential) will be one of the next steps needed to obtain a more fundamental understanding. \\ \\
In order to be certain if a configuration really represents a true attractor the eigenvalues related to this configuration must be considered. An attractor can have both positive and negative eigenvalues. They are called Lyapunov exponents since we follow them 'around in space' (in this case the 3D Jeans parameter space. They are not calculated in a point). This is a possible next step to test attractors further. In [15] the flow towards the attractor found in [4] is examined and it is found that the further away ICs are from this attractor, the faster they will flow towards it. Also the larger the perturbations the larger the flow rate towards the attractor. Structures then slows down as they approach the attractor and are never seen to cross it (in a way that stars would when flowing towards the main sequence attractor found on a HR diagram) thus indicating a damped oscillator type system which could potentially be described by linear equations from which the Lyaponov exponents might be found and examined. It still remains to find the attractor in cosmological halos, which might have to do with the fact that $\beta$ is different along different directions as well as structures not being totally spherical [9]. To see my analysis codes, please visit the following web-adress: \\
\url{https://bitbucket.org/dark_knights/darkmatterproject}.