\begin{thebibliography}{99}

\bibitem{}  Martin Sparre \emph{Eddington-Code developed for Sparre \& Hansen 2012, JCAP, 07,042 and Sparre \& Hansen 2012, JCAP,10,049},  available at \\
\url{https://github.com/martinsparre/Eddington-Code}.

\bibitem{}  Martin Sparre \emph{Osipkov-Merritt-Code developed for Sparre \& Hansen 2012, JCAP, 07,042 and Sparre \& Hansen 2012, JCAP,10,049},  available at \\
\url{https://github.com/martinsparre/OsipkovMerritt}.

\bibitem{} J. Binney and S. Tremaine, Galactic Dynamics: Second Edition, Princeton University Press, Princeton U.S.A. (2008).

\bibitem{} S.H. Hansen, D. Juncher and M. Sparre, An attractor for dark matter structures, arXiv:1005.1643 [INSPIRE].

\bibitem{} S.H. Hansen and B. Moore, A Universal density slope - velocity anisotropy relation for relaxed structures, New Astron. 11 (2006) 333 [astro-ph/0411473] [INSPIRE].

\bibitem{} S.H. Hansen and J. Stadel, The velocity anisotropy-density slope relation, JCAP 05 (2006) 014 [astro-ph/0510656] [INSPIRE].

\bibitem{} S.H. Hansen and M. Sparre, The behaviour of shape and velocity anisotropy in dark matter haloes, arXiv:1210.2392 [astro-ph.CO].

\bibitem{} A. Meza and N. Zamorano, Numerical stability of a family of Osipkov-Merrit models, arXiv:astro-ph/9707004

\bibitem{} M. Sparre and S.H. Hansen, Asymmetric velocity anisotropies in remnants of collisionless mergers, arXiv:1205.1799 [astro-ph.CO]

\bibitem{} V. Springel, The Cosmological simulation code GADGET-2, Mon. Not. Roy. Astron. Soc. 364 (2005) 1105 [astro-ph/0505010] [INSPIRE].

\bibitem{} V. Springel, High perfomance computing and numerical modeling, Lecture 1:
Collisional and collisionless N-body dynamics, 43rd Saas-Fee Course, Star formation in galaxy evolution: connecting numerical models to reality (11-16 March 2013)

\bibitem{} M. Vogelsberger et. al., Dwarf galaxies in CDM and SIDM with baryons: observational probes of the nature of dark matter, arXiv: 1405.5216

\bibitem{} S. Weinberg, Gravitation and Cosmology: Principles and Applications of the General Theory of Relativity, 1972.

\bibitem{} S. Carroll, Spacetime and Geometry: An Introduction to General Relativity.

\bibitem{} J.A. Barber, H. Zhao, X. Wu and S.H. Hansen, Stirring N-body systems: Universality of end states, arXiv:1204.2764 [INSPIRE].

\bibitem{} B. Ryden, Introduction to Cosmology.

\bibitem{} S.H. Hansen, Dark matter density profiles from the jeans equation, Mon. Not. Roy. Astron. Soc. 352 (2004) L41 [astro-ph/0405371] [INSPIRE].

\bibitem{} S.H. Hansen, Might we eventually understand the origin of the dark matter velocity anisotropy?, Astrophys. J. 694 (2009) 1250 [arXiv:0812.1048] [INSPIRE].

\bibitem{} S.H. Hansen and M. Sparre, A derivation of (half) the dark matter distribution function, arXiv:1206.5306 [INSPIRE].

\bibitem{} A.S. Eddington, The distribution of stars in globular clusters, Mon. Not. Roy. Astron. Soc. 76 (1916) 572.

\bibitem{} L. Hernquist, An analytical model for spherical galaxies and bulges, Astrophys. J. 356 (1990) 359.

\bibitem{} Power et. al., The inner structure of $\Lambda$CDM haloes - I. A numerical convergence study (2003)

\bibitem{} G. C. Rasmussen, Computational Astrophysics: Structures of dark matter halos around galaxy clusters, exam report for the course 'Computational Astrophysics' at NBI, UCPH (2015).

\end{thebibliography}