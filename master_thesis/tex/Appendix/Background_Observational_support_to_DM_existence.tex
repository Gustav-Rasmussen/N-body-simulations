\section{Background: Observational support to DM existence}
\textit{'Somewhere, something incredible is waiting to be known.'} \\
-- Carl Sagan \\ \\

DM was postulated by Jan Oort (1932) who studied orbital velocities of stars in the MW. Fritz Zwicky found there was missing mass (1933) when studying the orbital velocities of galaxies in the COMA cluster. The COMA cluster of galaxies (Abell 1656) is the home of more than one thousand galaxies. It is located inside the Coma Supercluster as a part of the Coma Berenices constellation. COMA is about $321 \cdot 10^6 $ light years away from earth. Horace W. Babcock studied galaxy rotation curves (1939) paving the road for Vera Rubin who postulated the existence of dark matter (1968) from galaxy rotation curves. Evidence for the existence of dark matter include gravitational lensing of distant astronomical objects by galaxy clusters such as the Bullet cluster and the COMA cluster (both micro-lensing and macro-lensing) and galaxy rotation curves which show velocities of galaxies inside clusters as a function of their distance to the cluster center. There is a discrepancy between the predicted theoretical curve (with only baryonic matter) and the observed curve which is much larger-valued for large distances to the cluster center. This is not possible if the only matter present is the visible that we observe, consider the virial theorem, $ K = \frac{1}{2} \cdot U $ where K and U are the kinetic and gravitational potential energy respectively. This is the simplified form of the theorem where the moment of inertia is neglected and thermal as well as magnetic energy is not considered. Another line of evidence for something extra is found in the CMB (more specifically, in the pattern of anisotropies in the CMB). Also we have the effects from Baryon Acoustic Oscillations (BAO); Basically it is periodic, regular fluctuations in the baryonic matter density which originated from the coupling (due to Thomson scattering) of free electrons and photons before recombination, that resulted in a distinct pattern of oscillations in the baryon and temperature power spectra. It acts as a standard ruler (of the order $490\cdot 10^6 $ lyr today) for length scales in cosmology, which can be estimated by astronomical surveys such as the Sloan Digital Sky Survey (SDSS). It is mainly used to research dark energy, but also requires constraining cosmological parameters such as the dark matter density. Newest Lyman $\alpha$ forest data exclude WDM models (Dwarf galaxies would be affected significantly). Finally should be mentioned the temperature distribution of hot gas in galaxies and galaxy clusters which point the way toward DM as well.