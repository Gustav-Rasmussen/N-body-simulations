\section{The $\Lambda$CDM Model and DM candidates}
In the framework of the currently most accepted cosmological model known as the $\Lambda$CDM, where the universe is dominated by dark energy parameterized by a cosmological constant $\Lambda$, and the matter content dominated by Cold Dark Matter (CDM), The prediction of the mechanism for structure formation is a so-called 'bottom-up' hierarchical formation of stars, planets, open and closed globular clusters, small galaxies gradually accreting and merging with other galaxies thereby growing in size and complexity, thus giving rise to the formation of galaxy clusters, and superclusters (clusters of galaxy clusters). The baryonic matter starts clumping together under the gravitational collapse that is invoked by the overdense regions of dark matter. Therefore, each galaxy cluster is assumed to be surrounded by massive halos of dark matter. It is these very halos which is simulated and studied in more detail, which hopefully can add understanding to the nature of dark matter which is a crucial step towards understanding structure formation amongst other physical phenomena. At this point it is still unknown what the dark matter consists of. MACHOs (MAssive Compact Halo Objects) can be black holes, brown dwarfs or white dwarfs. These have been excluded as DM candidates from half an earth mass up to 30 solar masses. WIMPs (Weakly Interacting Massive Particles) In this study it is assumed that the dark matter particles consist of primarily collisionless particles, but other people are working on models which include collisional dark matter,e.g. SIMPs (Strongly Interacting Massive Particles) 'Cold' dark matter means that the velocities of the dark matter particles on average are much less than the speed of light in vacuum. 
Other types of dark matter have also been proposed such as warm dark matter and hot dark matter. WIMPs are quite heavy; somewhere between 1 GeV (mass of a single proton), and 1 TeV. They are therefore relatively slow, or ‘cold’. With SUperSYmmetric (SUSY) particles which are still speculative, annihilation processes has also been theorized to explain certain phenomena within the realms of dark matter, such as radio emission signals from the center of our Milky Way galaxy. Particles such as axions and sterile neutrinos, chameleon particles, Wimpzillas etc. are also considered. They all have different locations along an energy axis, and this axis is slowly being ruled out more and more, section by section, in the hopes of discovering the real particle along the way. The name ‘sterile’ neutrino refers to the fact that this fourth, theoretical particle would lack flavor and therefore not feel the weak force as opposed to the ordinary first three generations of neutrinos ($\nu_e$, $\nu_{\mu}$ and $\nu_{\tau}$)